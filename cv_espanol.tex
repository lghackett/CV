%%%%%%%%%%%%%%%%%%%%%%%%%%%%%%%%%%%%%%%%%
% Medium Length Professional CV
% LaTeX Template
% Version 2.0 (8/5/13)
%
% This template has been downloaded from:
% http://www.LaTeXTemplates.com
%
% Original author:
% Trey Hunner (http://www.treyhunner.com/)
%
% Important note:
% This template requires the resume.cls file to be in the same directory as the
% .tex file. The resume.cls file provides the resume style used for structuring the
% document.
%
%%%%%%%%%%%%%%%%%%%%%%%%%%%%%%%%%%%%%%%%%

%----------------------------------------------------------------------------------------
%	PACKAGES AND OTHER DOCUMENT CONFIGURATIONS
%----------------------------------------------------------------------------------------

\documentclass{resume} % Use the custom resume.cls style
\usepackage[spanish]{babel}
\selectlanguage{spanish}
\usepackage[utf8]{inputenc}
\usepackage[left=0.75in,top=0.6in,right=0.75in,bottom=0.6in]{geometry} % Document margins
\newcommand{\tab}[1]{\hspace{.2667\textwidth}\rlap{#1}}
\newcommand{\itab}[1]{\hspace{0em}\rlap{#1}}
\name{Lucy Hackett} % Your name
\address{Ciudad de México, México} % Your address
%\address{123 Pleasant Lane \\ City, State 12345} % Your secondary addess (optional)
\address{(+521)~55~7129~9274 \\ lucyhackett23@gmail.com} % Your phone number and email

\begin{document}

%----------------------------------------------------------------------------------------
%	EDUCATION SECTION
%----------------------------------------------------------------------------------------

\begin{rSection}{Educación}

{\bf CIDE (Centro de Investigación y Docencia Económicas)} \hfill {\em agosto 2016 - junio 2018} 
\\ Ciudad de México
\\ Maestría en Economía \hfill {Promedio: 9.7/10}
\\ Premio de Mejor Tesina Aplicada \\
Tesina: Trabajando para un cambio: El efecto de la participación laboral femenina sobre la fecundidad \\
Asesora: Fernanda Marquez Padilla  \\
%Member of Eta Kappa Nu \\
%Member of Upsilon Pi Epsilon \\
\vspace{0.2cm}

{\bf Honors College, Universidad de Oregon} \hfill {\em sept. 2011 - junio 2015} 
\\ Doble carrera: Economía, Letras hispanas \hfill {Promedio: 4.1/4}
\\ Especialidad en matemáticas \hfill \textit{Summa cum laude}
\\ Premio Presidencial para Investigación Original \\
Tesis: English Programs in Oregon: Helping or Hurting English
Language Learners?  \\
Asesor: Glenn Waddell  \\

\vspace{0.2cm}

{\bf UNAM} \hfill {\em Fall 2014} 
\\ Intercambio estudiantil en la Facultad de Filosofía y Letras \hfill {Promedio: 9.8/10}
\\ Ciudad de México

\end{rSection}
%----------------------------------------------------------------------------------------
%	TECHNICAL STRENGTHS SECTION
%----------------------------------------------------------------------------------------

\begin{rSection}{Habilidades técnicas, idiomas}

\begin{tabular}{ @{} >{\bfseries}l @{\hspace{6ex}} l @{\hspace{6ex}}  @{} >{\bfseries}l @{\hspace{6ex}} l }
Avanzado        & Stata, R, \LaTeX , Excel, Git & Inglés & Nativo \\
Intermedio    & Java, Python  & Español & Con fluidez  \\
\end{tabular}

\end{rSection}

%----------------------------------------------------------------------------------------
%	WORK EXPERIENCE SECTION
%----------------------------------------------------------------------------------------
\begin{rSection}{Experiencia}

\begin{rSubsection}{LNPP (Laboratorio Nacional de Políticas Públicas)}{agosto 2018-presente}{Analista}{Ciudad de México}
\item Trabajo en un equipo de investigadores que combinan el análisis económico con la modelación computacional y econométrico para analizar cambios en la política pública \textit{ex-ante}
\item Programación en R, Python, Java y Stata para la construcción y validación de modelos 
\item Doy cursos para servidores públicos para la implementación y evaluación de programas sociales
\end{rSubsection}

%------------------------------------------------

\begin{rSubsection}{División de Economía: CIDE}{agosto 2018-presente}{Asistente de Investigación}{}
\item Busco, limpio y analizo datos para dos profesoras de la División de Economía (Dra. Luz MArina Arias y Dra. Fernanda Marquez Padilla)
\item Tengo experiencia trabajando con bases de datos diversas, incluyendo datos micto de consultas médicas, datos poblacionales, administrativos, de encuesta y geoestadísticos.
\end{rSubsection}

\begin{rSubsection}{División de Economía: CIDE}{agosto 2017-diciembre 2018}{Laboratorista}{}
\item Diseñé e impartí el laboratorio de Macroeconomía Avanzada (nivel maestría) y Econometría II (nivel maestría)
\item Mis responsabilidades incluyeron sesiones cada semana, la creación y evaluación de tareas, y apoyo a los alumnos fuera de clase 
\end{rSubsection}

\begin{rSubsection}{División de Economía: CIDE}{julio 2018, 2019}{Profesora}{}
\item Planeé e impartí el curso propedéutico de Macroeconomía para la Maestría en Economía 
\item Di 30 horas de clase y coordiné con un laboratorista para coordinar tareas y sesiones de laboratorio. Creé ejercicios y examenes
\item La materia del curso incluyó introducción a conceptos macroeconómicos, optimización finita e infinita, y macro microfundamentada 
\end{rSubsection}

\begin{rSubsection}{Facultad de Economía: Univ. de Oregon}{2014}{Asistente de Investigación}{Eugene, OR}
\item Ayudé al Dr. Ben Hansen con preparación de datos y de manuscritos para publicación
\end{rSubsection}

\end{rSection}

%	EXAMPLE SECTION
%----------------------------------------------------------------------------------------

\begin{rSection}{Logros académicos} \itemsep -2pt
\item Ranqueada 1\textsuperscript{ra} de mi generación de Maestría (13 alumnos) 
\item Premio para mejor tesina apliacada, Maestría en Economía
\item Premio presidencial para investigación original, Licenciatura
\item \textit{Summa cum laude} en licenciatura 
\item Phi Beta Kappa

\end{rSection}

%----------------------------------------------------------------------------------------
\begin{rSection}{Cursos}
\itab{\textbf{Evaluación de Programas}} \tab{}  \tab{\textbf{Intro la a Macroeconomía}}
\\ \itab{35 alumnos (servidores públicos) } \tab{}  \tab{60 alumnos (curso filtro para la ME)}
\\ \itab{24 horas} \tab{}  \tab{30 horas} 
\\ \itab{Temas: RCT's, métodos cuasiexperimentales,} \tab{}  \tab{Temas: optimización en 2, $\infty$ periodos, } 
\\ \itab{la crítica cualitativa} \tab{} \tab{into a la teoría monetaria, economía abierta}
\\ \itab{} \tab{} \tab{Evaluación global de alumnos: 9.8/10 }
\\ \itab{} \tab{} \tab{(Promedio para profesores: 9.4)}
% \\ \itab{Process Control (ongoing)} \tab{} \tab{Electrodynamics}

\end{rSection}



\begin{rSection}{Conferencias}

\begin{rSubsection}{LACEA Health Workshop}{nov. 2018}{Presenté working paper}{Bogota, Colombia}
\item Presenté el working paper \textit{Working for Change: The effect of female labor force participation on fertility}, coautoreado con Fernanda Marquez Padilla
\end{rSubsection}


\end{rSection}

\end{document}
