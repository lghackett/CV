\documentclass[10pt, oneside]{article}
\usepackage{geometry}

%\usepackage[T1]{fontenc}
\usepackage{geometry}
\usepackage{changepage} 
\usepackage[pdftex]{graphicx}
\usepackage{multicol}
\usepackage{ gensymb }
\usepackage[english,spanish]{babel}
\usepackage[utf8]{inputenc}
\usepackage{enumitem} 
\setlist{leftmargin=3mm}

\usepackage{dcolumn}

\makeatletter 

\newlength\tdima
\newcommand\tabfill[1]{%
      \setlength\tdima{\linewidth}%
      \addtolength\tdima{\@totalleftmargin}%
      \addtolength\tdima{-\dimen\@curtab}%
      \parbox[t]{\tdima}{#1\ifhmode\strut\fi}}

\newcommand\mytabs{\hspace*{1cm}\=\hspace{1cm}\=\hspace{1cm}\=\hspace{1cm}\=\hspace{1cm}\=\hspace{1cm}\=\hspace{1cm}\=\hspace{1cm}\=\hspace{1cm}\=\hspace{1cm}}
\newenvironment{mysec}[1][\mytabs]
  {\begin{tabbing}#1\kill\ignorespaces}
  {\end{tabbing}}
 
\makeatother
  
\usepackage{titlesec}

\newcommand{\lyxrightaddress}[1]{
\par {\raggedleft \begin{tabular}{l}\ignorespaces
#1
\end{tabular}
\vspace{1.4em}
\par}
}

\geometry{letterpaper}
\geometry{verbose,tmargin=1.3cm,bmargin=1.3cm,lmargin=1.2cm,rmargin=1.2cm}
% Activate to surpress page number on first page:
\thispagestyle{empty}

\usepackage{soul}
\usepackage{times}

\usepackage{ifthen,xcolor}
\newlength{\tabcont}
\newcommand{\tab}[1]{%
\settowidth{\tabcont}{#1}%
\ifthenelse{\lengthtest{\tabcont < .25\linewidth}}%
{\makebox[.25\linewidth][l]{#1}\ignorespaces}%
}%

\usepackage{hyperref}
\hypersetup{colorlinks=true, urlcolor=black}
 
\urlstyle{same}

\usepackage{titlesec}
\titleformat{\subsection}
  {\large\scshape\bfseries}
  {\thesubsection}{1em}{}

\setlength{\columnsep}{0pt}
\hypersetup{
     colorlinks   = true,
     urlcolor    = blue
}
\begin{document}

%%%%%%%%%%%%%%%%%%%%%%%%%%%%%%%%%%%%%%%%%%%%%%%%%%%%%%%%%%%%%%%%%%%%%%%%%%%%%%%%%%%%%%%%%%%%%%%%%%%%%%%%%%%%

\pagenumbering{gobble}

\begin{center}

\begin{picture}(1000,1)
    \put(0,-20){\includegraphics[width=\textwidth]{img/formalheader.png}}
\end{picture}

\vspace{10mm}
{ University of California, Berkeley \hfill Department of Agricultural and Resource Economics} \\\hrule  \vspace{10mm}
{\Large \textbf{Lucy Hackett}} \\
\end{center} 

%%% CONTACT INFORMATION
\begin{minipage}[t]{0.1\linewidth}
\textbf{Contact \\ Information}
\end{minipage}\hspace{0.05\linewidth}
\begin{minipage}[t]{0.8\linewidth}
Giannini 241, UC Berkeley \\
\href{mailto:lucy_hackett@berkeley.edu}{lucy\_hackett@berkeley.edu} \\
\href{https://lghackett.github.io}{https://lghackett.github.io} \\
+1-(510)-326-3451
\end{minipage}\vspace{5mm}

%%% DOCTORAL STUDIES
\begin{minipage}[t]{0.1\linewidth}
\textbf{Doctoral Studies}
\end{minipage}\hspace{0.05\linewidth}
\begin{minipage}[t]{0.8\linewidth}
University of California, Berkeley\\
PhD, Agricultural and Resource Economics, Expected completion May 2026 \\
\textsc{Dissertation:}  ``Essays in Environmental and Urban Economics'' \\ 
~\\ 
\textsc{Primary Fields:} Environment, Development Economics \\
\textsc{Secondary Fields:} Urban/Spatial Economics \\
\vspace{0.3cm} 

\begin{minipage}[t]{0.5\linewidth}
\begin{itemize}[noitemsep,nolistsep]
\item[] \underline{Associate Professor Marco Gonzalez-Navarro} \\
\href{mailto:marcog@berkeley.edu}{marcog@berkeley.edu} \\ 
+1 (510) 390-4720 \\
Department of Agricultural
\item[] \hspace{4mm} \& Resource Economics \\
\item[] \underline{Professor Lucas Davis} \\
\href{mailto:lwdavis@berkeley.edu}{lwdavis@berkeley.edu} \\
%+1 (999) Phone \\
Haas School of Business 
\end{itemize}
\end{minipage}
\begin{minipage}[t]{0.4\linewidth}
\begin{itemize}[noitemsep,nolistsep]
    \item[] \underline{Assistant Professor Kirill Borusyak} \\
\href{k.borusyak@berkeley.edu}{k.borusyak@berkeley.edu} \\ 
%+1 (999) Phone \\
Department of Agricultural
\item[] \hspace{4mm} \& Resource Economics\\\\ 
\item[] \underline{Professor Joseph Shapiro} \\
\href{mailto:joseph.shapiro@berkeley.edu}{joseph.shapiro@berkeley.edu} \\ 
%+1 (999) Phone \\
Department of Agricultural
\item[] \hspace{4mm} \& Resource Economics
\end{itemize}
\end{minipage}
\end{minipage}\vspace{5mm}

%%% PLACEMENT OFFICERS
\begin{minipage}[t]{0.1\linewidth}
\textbf{Placement\\ Officers}
\end{minipage}\hspace{0.04\linewidth}
\begin{minipage}[t]{0.9\linewidth}~\vspace{-9mm}
\begin{multicols}{3}
\begin{itemize}[noitemsep,nolistsep]
\item[] \underline{Professor Sofia Villas-Boas} \\
\href{mailto:sberto@berkeley.edu}{sberto@berkeley.edu} \\
+1 (510) 409-4341 \\
\item[] \underline{Professor Marco Gonzalez-Navarro} \\
\href{mailto:marcog@berkeley.edu}{marcog@berkeley.edu} \\
+1 (510) 390-4720 \\
\item[] \underline{Diana Lazo} \\
\href{mailto:lazo@berkeley.edu}{lazo@berkeley.edu} \\
+1 (510) 642-3345 
\end{itemize}
\end{multicols}
\end{minipage}\vspace{5mm}

%%% PRIOR EDUCATION
\begin{minipage}[t]{0.1\linewidth}
\textbf{Prior \\ Education}
\end{minipage}\hspace{0.05\linewidth}
\begin{minipage}[t]{0.8\linewidth}
\begin{mysec} 
\textbf{University of Oregon} \>\>\>\>\> B.Sc. in Economics and Spanish Literature \` 2015 
\end{mysec} 
\begin{mysec} 
    \textbf{CIDE} \>\>\>\>\> Master's Degree in Economics \` 2018
    \end{mysec} 
\end{minipage}\vspace{5mm}

%%% TEACHING
\begin{minipage}[t]{0.1\linewidth}
\textbf{Teaching}
\end{minipage}\hspace{0.05\linewidth}
\begin{minipage}[t]{0.8\linewidth}
\begin{mysec} 
    \textbf{UC Berkeley} \>\>\>Instructor, \emph{Microeconomics}  \` 2022, 2023 \\
    \>\>\> Master's in Development Practice
    \end{mysec}
\begin{mysec} 
\textbf{UC Berkeley} \>\>\>Teaching Assistant, \emph{Graduate Econometrics}  \` 2021, 2022, 2023 \\
\>\>\> PhD Agricultural and Resource Economics
\end{mysec}
\begin{mysec} 
\textbf{UC Berkeley} \>\>\>Instructor, \emph{Training for New Graduate Student Instructors}  \` 2021, 2022, 2023 \\
\>\>\> UC Berkeley 
\end{mysec}
\begin{mysec} 
    \textbf{CIDE} \>\>\>Instructor, \emph{Graduate Macroeconomics}  \` 2020 \\
    \>\>\> Master's in Economics 
\end{mysec}
\begin{mysec} 
    \textbf{CIDE} \>\>\>Instructor, \emph{Macroeconomics}  \` 2019 \\
    \>\>\> Economics (undergraduate)
\end{mysec}
\begin{mysec} 
    \textbf{CIDE} \>\>\>Teaching Assistant, \emph{Macroeconomics}  \` 2017, 2018, 2019 \\
    \>\>\> Economics (Graduate and undergraduate)
\end{mysec}
\end{minipage}\vspace{5mm}

%%% LANGUAGES
\begin{minipage}[t]{0.1\linewidth}
\textbf{Languages}
\end{minipage}\hspace{0.05\linewidth}
\begin{minipage}[t]{0.8\linewidth}
English (native), Spanish (fluent)
\end{minipage}\vspace{5mm}

%%% GRANTS, FELLOWSHIPS, AWARDS
\begin{minipage}[t]{0.1\linewidth}
\textbf{Grants, Fellowships, and Awards}
\end{minipage}\hspace{0.05\linewidth}
\begin{minipage}[t]{0.8\linewidth}
\begin{mysec} 
	2025 \>\>\tabfill{UC Dissertation Fellowship (\$37,000)} \\
    2025 \>\>\tabfill{CEGA Development Economics Challenge Grant (\$5,000)} \\
    2024 \>\>\tabfill{Clausen Center for International Trade Research Grant (\$5,000)} \\
    2024 \>\>\tabfill{Fisher Center for Real Estate Research Grant  (\$15,000)} \\
	2023 \>\>\tabfill{Fisher Center for Real Estate Research Grant  (\$15,000)} \\
    2023 \>\>\tabfill{Giannini Foundation Minigrant  (\$35,000)} \\
    2022 \>\>\tabfill{ARE travel grant  (\$2,000)} \\
    2021 \>\>\tabfill{Tinker Field Research Grant (\$2,000)}\\ 
    2021 \>\>\tabfill{Outstanding Graduate Student Instructor Award}\\ 
    2018 \>\>\tabfill{Best applied thesis, Class of 2018, CIDE}\\ 
    2015 \>\>\tabfill{Phi Beta Kappa}\\ 
    2015 \>\>\tabfill{Oregon 6}
\end{mysec} 
\end{minipage}\vspace{5mm}

%%% RESEARCH PAPERS
\begin{minipage}[t]{0.1\linewidth}
\textbf{Research \\ Papers}
\end{minipage}\hspace{0.05\linewidth}
\begin{minipage}[t]{0.8\linewidth}
\textbf{``Land Subsidence: Environmental risk in housing markets in Mexico City''\\ 
(\textsc{Job Market Paper})}\\ 
 with Carolina Rodriguez-Zamora \\

We study the costs of and the housing market response to subsidence, the sinking of land areas due to groundwater over-extraction, in Mexico City. Subsidence is a prevalent and worsening phenomenon worldwide, considered in the civil engineering literature to be a major risk to infrastructure in affected urban areas. We use a repeat bank appraisal approach to estimate the costs of subsidence through the housing market, quantifying both damages to private housing units and public infrastructure. Despite our finding that subsidence imposes substantial costs, which are driven by increased likelihood of structural issues and urban flooding, we find that housing development is drawn to sinking plots and away from sinking neighborhoods. An equilibrium model of the housing market rationalizes these findings, highlighting that sunk homes have a low opportunity cost of redevelopment, but that this incentive is distorted if residents face information frictions about future subsidence. We explore this using novel survey evidence on subsidence and beliefs in the city, documenting the presence of substantial frictions in evaluating the risk of future sinking. We use these findings together with our structural model to analyze the welfare gains from policies that address subsidence by reducing groundwater pumping, or alternatively that mitigate information frictions in the housing market.

% \textbf{``Title.''} \emph{Revision requested by \emph{Journal}. Date YYYY.} \href{address}{\color{blue}{text}}. \\
% Abstract
\end{minipage}\vspace{5mm}


%%% RESEARCH IN PROGRESS
\begin{minipage}[t]{0.1\linewidth}
\textbf{Research in Progress}
\end{minipage}\hspace{0.05\linewidth}
\begin{minipage}[t]{0.8\linewidth}
\textbf{``Estimating the Gains from Water Trade: A Systematic
Evaluation of Modeling Considerations''} with Nell Green Nylen, Ellen Bruno, Andrew Ayers, Michael Kiparsky, Josué Medellín-Azuara, and Sarah Null \\ 
\emph{Draft available upon request.} \\

The gains from water trading can vary significantly depending on local conditions as well as the specifics of market design and implementation. However, models of water trading necessarily rely on assumptions that simplify the social, institutional, and environmental landscape within which a water market operates. We systematically evaluate peer-reviewed papers that estimate the gains from water trading to assess how models of water markets  take this local context into account. Our results demonstrate that whether and how models incorporate key considerations varies widely, with implications for the accuracy of results. We find that estimates of the economic impacts of water trading in the published literature are more likely to consider distributional effects and incorporate features of the legal and regulatory environment than to account for third-party impacts, transaction costs, the consequences of trading for the economy at large, or the administrative costs associated with setting up and operating a market. Understanding what features a model takes into account is important for interpreting its policy implications. Researchers modeling the gains from trade could better support local decision makers by explicitly articulating their models’ capabilities and limitations.\\ 

\textbf{``Differential subsidence, damages and fragility: Evidence from a systematic analysis in Mexico City''} with Enrique Fernández-Torres\\ 

Understanding the structural vulnerability of buildings and public infrastructure to differential subsidence is crucial to evaluating the risks and costs that subsidence poses in urban areas. We combine novel estimates of plot-specific differential subsidence in Mexico City with a representative resident survey of structural issues in private residents and public infrastructure to estimate structural fragility curves and damage thresholds. We then extrapolate these findings from micro-data to a city-wide analysis, calculating damages and vulnerability at a city block level. \\ 
\end{minipage}

%%% TALKS
\begin{minipage}[t]{0.1\linewidth}
\textbf{Talks}
\end{minipage}\hspace{0.05\linewidth}
\begin{minipage}[t]{0.8\linewidth}
\begin{mysec} 
    2025 \>\> \tabfill{AERE Summer meeting} \\
	2024 \>\> \tabfill{LACEA Urban Workshop} \\
    2024 \>\> \tabfill{\textit{SobreMéxico} Conference,  Universidad Iberoamericana}\\ 
    2024 \>\> \tabfill{UC Berkeley Development workshop} \\
    2023, 2024 \>\> \tabfill{UC Berkeley Trade workshop} \\
\end{mysec} 
\end{minipage}\vspace{5mm}

%%% REFEREEING
% \begin{minipage}[t]{0.1\linewidth}
% \textbf{Refereeing}
% \end{minipage}\hspace{0.05\linewidth}
% \begin{minipage}[t]{0.8\linewidth}
% \emph{Journal of Excellent Papers}
% \end{minipage}\vspace{5mm}

%%% ACTIVITIES
\begin{minipage}[t]{0.1\linewidth}
\textbf{Activities}
\end{minipage}\hspace{0.05\linewidth}
\begin{minipage}[t]{0.8\linewidth}
\begin{mysec}
    2024 \>\> Berkeley Economists for Equity (BEE) Mentor\\ 
    2023 \>\> Department of Agricultural and Resource Economics Graduate Admissions Committee \\
    2022 \>\> Department of Agricultural and Resource Economics Search Committee \\
    2020 - 2022 \>\> Department of Agricultural and Resource Economics DEI committee \\


\end{mysec}
\end{minipage}\vspace{5mm}


%%% PRIOR EMPLOYMENT
\begin{minipage}[t]{0.1\linewidth}
\textbf{Prior \\ Employment}
\end{minipage}\hspace{0.05\linewidth}
\begin{minipage}[t]{0.8\linewidth}
\textbf{UC Berkeley}, Graduate Student Researcher (Marco Gonzalez-Navarro) \hfill 2023-2025 \\
\textbf{UC Berkeley}, Graduate Student Researcher (Ethan Ligon) \hfill 2022 \\
\textbf{UC Berkeley}, Graduate Student Researcher (Ellen Bruno) \hfill 2023 \\
\textbf{National Laboratory for Public Policy (LNPP)}, Analyst \hfill 2018-2020 \\
\end{minipage}\vspace{5mm}

%%% ANY OTHER ITEMS
\begin{minipage}[t]{0.1\linewidth}
\textbf{}
\end{minipage}\hspace{0.05\linewidth}
\hspace{0.05\linewidth}
\begin{minipage}[t]{0.8\linewidth}


\end{minipage}\vspace{5mm}


\end{document}